\documentclass{beamer}
 
\usepackage[utf8]{inputenc}
 
 
%Information to be included in the title page:
\title{Hacking Medical Devices for Fun and Insulin: Breaking the Human 
SCADA System}
\subtitle{Analisi delle vulnerabilità del sistema CGM}
\author{Francesco Montelli}
\institute{CeSeNa}
\date{2017}
 
 
 
\begin{document}
 
\frame{\titlepage}

\begin{frame}
	\frametitle{Analisi delle vulnerabilità del sistema CGM}
	\begin{itemize}
		\item Ipotesi
		\item Analisi
		\item Esperimenti
		\item Considerazioni
		\item Conclusione
	\end{itemize}
\end{frame}
 
\begin{frame}
\frametitle{Ipotesi}
	\begin{itemize}
		\item Crittografia?
		\item Tipo di comunicazione
		\item Di che cosa ``si rende conto'' il sensore?
	\end{itemize}
\end{frame}

\begin{frame}
\frametitle{Analisi}
	\begin{itemize}
		\item Read the manual
			\begin{table}
			\centering
			\begin{tabular}{ll}
				Transmitter/Reciver Frequency & 402.142 MHz            \\
				Bandwidth                     & 300 KHz                \\
				Modulation                    & On-Off Key             \\
				Data Rate                     & 8192 bits/Sec          \\
				Total Packt                   & 76 bit                 \\
				Transmit Duty Cycle           & 9.28 ms evry 5 minutes                           
			\end{tabular}
			\end{table}
	\end{itemize}
\end{frame}

\begin{frame}
\frametitle{Analisi}
	\begin{itemize}
		\item FCC Recon Research
		\item Brevetti
		\item Smontare il CGM
			\begin{itemize}
				\item Nome del chip visibile (AMIS 52100M)
				\item Stesso chip usato in ambienti SCADA
			\end{itemize}
	\end{itemize}
\end{frame}

\begin{frame}
\frametitle{Esperimenti - How to Listen}
	\begin{itemize}
		\item Arduino
		\item Problemi
			\begin{itemize}
				\item Tanti registri (80+)
				\item Tante impostazioni
				\item Chip fuori produzione -> nessun supporto del produttore
			\end{itemize}
		\item Scoperte
			\begin{itemize}
				\item Nessun controllo di consistenza ai valori assegnati ai registri
			\end{itemize}
	\end{itemize}
\end{frame}

\begin{frame}
\frametitle{Esperimenti - CGM Signal Dissection}
	\begin{itemize}
		\item Modulazione OOK
		\item Segnale = 1, Nessun sengale = 0
		\item 8192 bits/sec * 9ms = circa 76bit 
	\end{itemize}
\end{frame}

\begin{frame}
\frametitle{Esperimenti - CGM Signal Dissection}
	\begin{itemize}
		\item Modulazione OOK
		\item Segnale = 1, Nessun sengale = 0
		\item 8192 bits/sec * 9ms = circa 76bit 
		\item Parametri richiesti per comunicare
			\begin{itemize}
				\item Preambolo (non menzionato nel datasheet)
				\item Sync Word: parola per autorizzare la comunicazione, impostata nel costruttore
				\ite CRC 
			\end{itemize}
		\item Direct Mode 
		\item Raccolta manuale dei pacchetti e realtiva decodifica
	\end{itemize}
\end{frame}

\begin{frame}
\frametitle{Considerazioni finali}
	\begin{itemize}
		\item Troppe combinazioni di impostazioni, impossibile avere un controllo fine 
		\item Nessuna documentazione utilizzabile
		\item Nessuna esprienza d'uso in questa modalità
	\end{itemize}
\end{frame}

\begin{frame}
\frametitle{Conclusioni - Rischi per la sicurezza}
	\begin{itemize}
		\item Replay Attack
		\item DOS
		\item Ogni attacco ideato risulterebbe in una semplice interruzione di servizio, nulla di grave considerando che si può tranquillamente misurare con il sangue
		\item Necessità di trovarsi vicino al bersaglio considerando la natura della trasmissione 
	\end{itemize}
\end{frame}



\end{document}

