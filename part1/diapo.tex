\documentclass{beamer}

\usepackage[utf8]{inputenc}
 
 
%Information to be included in the title page:
\title{Hacking Medical Devices for Fun and Insulin: Breaking the Human 
SCADA System}
\subtitle{Introduzione}
\author{Francesco Montelli}
\institute{CeSeNa}
\date{2017}
 
 
 
\begin{document}
 
\frame{\titlepage}

\begin{frame}
	\frametitle{Introduzione}
	\begin{itemize}
		\item Insulina
		\item Infusore vs. ``penne''
		\item CGM - Continuous Glucose Monitoring
		\item Conseguenze di un controllo non ottimale/errato
		\item Closed Loop/Automated Insulin Delivery
	\end{itemize}
\end{frame}
 
\begin{frame}
	\frametitle{Insulina}
	\begin{itemize}
	\item Ormone prodotto dal pancreas, fondamentale nel processo di glicosi
	\item In soggetti diabetici deve essere introdotta artificialmente nel corpo
	\end{itemize} 
\end{frame}

\begin{frame}
	\frametitle{Infusore vs. ``penne''}
	Infusore
	\begin{itemize}
		\item Controllo elettronico
		\item Maggiore precisione
		\item Permette il contrllo della basale attraverso infusioni continue di insulina ``rapida'' (atenza di mezz'ora, raggiunge il picco in due-quattro ore e la sua attività scompare dopo quattro-otto ore)
		\item Rischi di chetoacidosi per malfunzionamento
	\end{itemize}
	Penne
	\begin{itemize}
		\item Controllo manuale
		\item Occorre tenere traccia manualmente delle dosi
		\item Necessita di iniettare una singola dose di insulina ``lenta'' (latenza di una-due ore, picco di 6-12 ore e durata di 18-24 ore)
		\item Guasti quasi impossibili
	\end{itemize}
\end{frame}

\begin{frame}
\frametitle{CGM - Continuous Glucose Monitoring}
	\begin{itemize}
		\item Importanza di avere informazioni sull'andamento del livello di glucosio
		\item Sensore che periodicamente comunica i livelli letti ad una unita' di controllo
	\end{itemize}
\end{frame}

\begin{frame}
\frametitle{Conseguenze di un controllo non ottimale/errato}
	\begin{itemize}
		\item Ipoglicemia
		\item Iperglicemia
		\item Chetoacidosi
		\item Nefropatia - Danni ai reni
		\item Neuropatia - Danni al sistema nervoso autonomo
		\item Retinopatia - Danni alla retina
	\end{itemize}
\end{frame}

\begin{frame}
\frametitle{Closed Loop/Automated Insulin Delivery}
	\begin{itemize}
		\item Eliminare la parte umana dal sistema
		\item Maggiore comunicazione tra sensore e infusore
		\item Somministrazione automatica di insulina
	\end{itemize}
\end{frame}

\end{document}

